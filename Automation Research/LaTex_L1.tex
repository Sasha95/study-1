\documentclass[a4paper,10pt]{article}

\usepackage[utf8]{inputenc}
\usepackage[russian]{babel}
\usepackage{amsmath}
\usepackage{amssymb}
\usepackage{indentfirst}
\usepackage{ccaption}
\captiondelim {. }

\usepackage[
top = 1.5cm,
bottom = 2cm,
left = 2cm,
right = 1.5cm
]{geometry}

\begin{document}
\newcommand{\vect}[1]{\boldsymbol{\mathbf{#1}}}
\title{Теорема Гаусса}
\author{Гаусс К. \footnote{ПГУ. им. Т.Г. Шевченко}}
\maketitle
\begin{abstract}
В статье кратко изложены сведения о теореме Гаусса --- важной и полезной <<Штуке>>.
\end{abstract}
\section{Введение}
\emph{Теорема Гаусса (закон Гаусса)} --- один из основных законов электродинамики, входит в систему уравнений Максвелла. Выражает связь (а именно равенство с точностью до постоянного коэффициента) между потоком напряжённости электрического поля сквозь замкнутую поверхность произвольной формы и алгебраической суммой зарядов, расположенных внутри этой поверхности, деленной на электрическую постоянную $\varepsilon_0$. Применяется отдельно для вычисления электростатических полей.

Теорему Гаусса можно записать для:
\begin{itemize}
\item{электрической индукции}
\item{магнитной индукции}
\end{itemize}

Рассмотрим только первый случай.

\section{Формула для электрической индукции}
Для поля в диэлектрической среде электростатическая теорема Гаусса может быть записана в виде (\ref{eq:gauss_int}).
\begin{equation} \label{eq:gauss_int}
\Phi_{\vect{D}}\equiv\oint_S\vect{D}d\vect{S}=4\pi Q.
\end{equation}

В дифференциальной форме:
\begin{equation} \label{eq:gauss_diff}
\operatorname{div}\vect{D}\equiv\nabla\cdot\vect{D}=4\pi\rho,
\end{equation}

Все выражения записаны для единиц в системе СГС. За подробностями обращайтесь к \cite{LL}.

\section{Дивергенция}
В выражении (\ref{eq:gauss_diff}) используется дивергенция. Это векторный оператор, определяемый следующим образом:
\begin{equation} \label{eq:div_vec}
\operatorname{div}\vect{A}=\nabla\vect{F}=\lim_{n \to \infty}\frac{\Phi_F}{V},
\end{equation}

где $\Phi_F$ --- поток векторного поля.

В декартовых координатах:
\begin{equation} \label{eq:div_dec}
\operatorname{div}\vect{F}=\frac{\partial F_x}{\partial x}+\frac{\partial F_y}{\partial y}+\frac{\partial F_z}{\partial z}.
\end{equation}

\begin{thebibliography}{99}
\bibitem{LL} Ландау, Л.Д., Лифшиц Е.М. Электродинамика сплошных сред. --- М.:Физматлит, 2005.
\end{thebibliography}

\end{document} 